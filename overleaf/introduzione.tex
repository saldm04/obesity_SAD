\section{Introduzione}
\label{sec:dataset-introduzione}

In questo lavoro utilizziamo il dataset \emph{Estimation of Obesity Levels Based On Eating Habits and Physical Condition}, reso disponibile tramite l’archivio UCI Machine Learning Repository \cite{uci-obesity}. Il dataset raccoglie informazioni su individui provenienti da alcuni Paesi dell’America Latina (in particolare Messico, Perù e Colombia), con l’obiettivo di stimare il livello di obesità a partire da abitudini alimentari e condizioni fisiche.
\newline
\newline
Le osservazioni comprendono variabili relative a caratteristiche anagrafiche e fisiche (ad esempio età, altezza, peso), abitudini alimentari (frequenza di consumo di determinati cibi, numero di pasti giornalieri, spuntini), stili di vita (attività fisica, uso di dispositivi tecnologici, consumo di alcolici, abitudine al fumo) e contesto familiare. A partire da queste informazioni viene definita una variabile di risposta binaria, che distingue due gruppi di soggetti sulla base del loro livello di peso corporeo.
\newline
\newline
Il dataset è sbilanciato rispetto alla variabile target, in quanto uno dei due gruppi di interesse è rappresentato da una quota sensibilmente inferiore di osservazioni rispetto all’altro. Inoltre, una parte dei dati è stata generata sinteticamente mediante tecniche di oversampling (con l’ausilio di strumenti di data mining), mentre la restante porzione è costituita da risposte raccolte direttamente dagli utenti tramite una piattaforma web.
\newline
\newline
Nel complesso, il dataset non presenta valori mancanti ed è stato pensato per supportare diversi tipi di compiti di apprendimento automatico, quali classificazione, regressione e clustering. Le analisi statistiche ed esplorative presentate in questo elaborato verranno condotte utilizzando il software R, che offre strumenti avanzati per la gestione dei dati, la visualizzazione e l’applicazione di metodi di modellazione.
\newpage