\section{Introduzione al Dataset}
\label{sec:dataset-overview}

Il dataset utilizzato nel presente studio è composto da 2111 osservazioni e 16 variabili esplicative, alle quali si aggiunge la variabile di risposta binaria \texttt{target}. Tale variabile distingue due gruppi di individui: il valore \texttt{1} identifica i soggetti classificati come \emph{Overweight Level I}, mentre il valore \texttt{0} aggrega tutte le rimanenti categorie di peso corporeo definite nella versione estesa del dataset originale (Insufficient Weight, Normal Weight, Overweight Level II, Obesity Type I, Obesity Type II e Obesity Type III).
\newline
\newline
Il dataset nella versione originale contiene sette classi distinte di livello di obesità, etichettate sulla base dell’\emph{Indice di Massa Corporea} (BMI), calcolato tramite la formula:
\[
\text{BMI} = \frac{\text{Peso (kg)}}{\text{Altezza (m)}^2}
\]
e successivamente categorizzato secondo le linee guida dell’Organizzazione Mondiale della Sanità (OMS).
\newline
\newline
La distribuzione delle etichette mostra una forte asimmetria: soltanto 290 osservazioni appartengono alla classe minoritaria (\texttt{1}), pari a circa il 13.7\% del totale, rendendo il dataset sensibilmente sbilanciato.
\newline
Il dataset è pensato per essere utilizzato in molteplici compiti di data mining, tra cui classificazione, predizione, clustering e analisi delle associazioni, grazie alla presenza di attributi sia numerici sia categorici.

\subsection{Features}
Le variabili incluse nel dataset coprono aspetti anagrafici, comportamentali e relativi alle abitudini alimentari e allo stile di vita degli individui. La selezione delle feature deriva da un’analisi di letteratura sui fattori riconosciuti come associati all’insorgenza dell’obesità e ai rischi cardiovascolari. Di seguito si riporta un riepilogo delle feature, coerente con la struttura del questionario utilizzato per la raccolta dati:

\begin{itemize}
\item \textbf{Gender} (Categorica): genere dell’individuo; valori: \emph{Female}, \emph{Male}.
\item \textbf{Age} (Continua): età in anni.
\item \textbf{Height} (Continua): altezza in metri.
\item \textbf{Weight} (Continua): peso in chilogrammi.
\item \textbf{family\_history\_with\_overweight} (Binaria): presenza di casi di sovrappeso in famiglia; valori: \emph{yes}, \emph{no}.
\item \textbf{FAVC} (Binaria): consumo frequente di cibi ad alto contenuto calorico; valori: \emph{yes}, \emph{no}.
\item \textbf{FCVC} (Continua): frequenza di consumo di verdure durante i pasti; valori: \emph{never - 0}, \emph{sometimes - 1}, \emph{always - 2}.
\item \textbf{NCP} (Continua): numero di pasti principali giornalieri; valori: \emph{One - 1}, \emph{Two - 2}, \emph{Three - 3}, \emph{More than three - 4}.
\item \textbf{CAEC} (Categorica): consumo di cibo tra un pasto e l’altro; valori: \emph{no}, \emph{Sometimes}, \emph{Frequently}, \emph{Always}.
\item \textbf{SMOKE} (Binaria): abitudine al fumo; valori: \emph{yes}, \emph{no}.
\item \textbf{CH2O} (Continua): quantità d’acqua consumata quotidianamente; valori: \emph{Less than a liter - 1}, \emph{Between 1 and 2L - 2}, \emph{More than 2L - 3}.
\item \textbf{SCC} (Binaria): monitoraggio dell’assunzione calorica giornaliera; valori: \emph{yes}, \emph{no}.
\item \textbf{FAF} (Continua): frequenza dell’attività fisica settimanale; valori: \emph{I do not have - 0}, \emph{1 or 2 days - 1}, \emph{2 or 4 days - 2}, \emph{4 or 5 days - 3}.
\item \textbf{TUE} (Continua): tempo di utilizzo quotidiano di dispositivi tecnologici; valori: \emph{0-2 hours - 0}, \emph{3-5 hours - 1}, \emph{More than 5 hours - 2}.
\item \textbf{CALC} (Categorica): frequenza del consumo di alcolici; valori: \emph{no}, \emph{Sometimes}, \emph{Frequently}, \emph{Always}.
\item \textbf{MTRANS} (Categorica): mezzo di trasporto maggiormente utilizzato; valori: \emph{Automobile}, \emph{Bike}, \emph{Motorbike}, \emph{Public\_Transportation}, \emph{Walking}.
\end{itemize}

\subsection{Processo di acquisizione dei dati}
La raccolta iniziale dei dati è avvenuta tramite una piattaforma web, attraverso la quale utenti anonimi provenienti da Colombia, Perù e Messico hanno compilato un questionario sulle proprie abitudini alimentari e condizioni fisiche. In questa fase sono stati acquisiti 485 record, sui quali è stato condotto un processo di pulizia che ha comportato la rimozione di dati mancanti, anomali o non validi, oltre a una fase di normalizzazione necessaria per la successiva applicazione di tecniche di data mining.
\newline
Poiché la distribuzione iniziale delle classi risultava fortemente sbilanciata (con alcune categorie dell’obesità scarsamente rappresentate), gli autori hanno applicato il metodo \emph{SMOTE} (Synthetic Minority Over-sampling Technique) tramite la piattaforma \emph{Weka} al fine di generare nuove osservazioni sintetiche nelle classi minoritarie.
\newline
Dopo il bilanciamento, il dataset ha raggiunto l’attuale dimensione di 2111 record. La presenza di un’ampia componente sintetica (pari al 77\% del totale) è un elemento dichiarato dagli autori e va considerata nell’interpretazione dei risultati, poiché consente un miglior addestramento dei modelli ma introduce una struttura meno rappresentativa rispetto a un campionamento interamente naturale. Il dataset finale non presenta valori mancanti.
\subsection{Preprocessing}
Dopo l’importazione del dataset in R viene effettuata una fase preliminare di preprocessing per verificarne l’adeguatezza rispetto alle analisi successive. Il dataset non presenta valori mancanti, caratteristica confermata anche dopo il caricamento, e le variabili categoriche vengono automaticamente convertite in oggetti di tipo \texttt{factor}.
Particolare attenzione è rivolta alle variabili \texttt{FAF}, \texttt{TUE}, \texttt{CH2O}, \texttt{FCVC} e \texttt{NCP}: pur essendo descritte come continue nella documentazione originale, esse derivano da scale a risposta chiusa con pochi livelli interi e vanno quindi interpretate come misure ordinali. Per preservare questa natura e mantenerne l’utilizzabilità nelle analisi descrittive, vengono trattate come variabili numeriche ordinali. Le variabili \texttt{Weight} e \texttt{Height}, invece, mantengono la loro forma continua originale senza necessità di trasformazioni.
\newpage
